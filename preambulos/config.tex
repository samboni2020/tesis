%\newcommand{\DoNotLoadEpstopdf}{}
\documentclass[a4paper,12pt,spanish,twoside,openright]{book}
\usepackage{amsmath}
\usepackage[utf8]{inputenc}
\usepackage[spanish]{babel} %escribir con acentos sin necesidad de comandos
\usepackage{fancyhdr}

\usepackage[scaled]{uarial} %%%Tipo de letra a usar en el documento
\renewcommand*\familydefault{\sfdefault} %% Only if the base font of the document is to be sans serif
\usepackage[T1]{fontenc} %%%Tipo de letra a usar en el documento

\usepackage{epic}
\usepackage{eepic}
\usepackage{threeparttable}
\usepackage{amscd}
\usepackage{textgreek}
\usepackage{here}
\usepackage{graphicx}
%\usepackage{epstopdf}
\usepackage{caption}
\usepackage{subcaption}
\expandafter\def\csname ver@subfig.sty\endcsname{}
\usepackage{lipsum}
\usepackage{lscape}
\usepackage{tabularx}
 \usepackage{multirow}
\usepackage{ltablex}
%\usepackage{fontspec}
\usepackage{xcolor, color, soul}
\usepackage{geometry}
\usepackage{longtable}
\usepackage{glossaries}
\usepackage{makeidx}
\usepackage{acronym}
\usepackage{thumbpdf}
\usepackage[hidelinks]{hyperref}
\usepackage{wrapfig,booktabs} %%Para tablas flotantes
\usepackage{nicematrix}
\setcounter{MaxMatrixCols}{30}
\usepackage{fontawesome, xcolor}

\usepackage{stix} %%%Para figuras geométricas, como pentagono
\usepackage{nicematrix} %%%%Para las matrices
\usepackage{colortbl}
\usepackage{tikzsymbols}
\usepackage{tikz}
\usetikzlibrary{shapes.geometric} % Load the library for geometric shapes
%%%%%Fin


\geometry{a4paper, top=20mm, left=30mm, right=10mm, bottom=20mm,
headsep=10mm, footskip=12mm}

\definecolor{mygreen}{RGB}{0,100,0}  %%%Se usa más adelante para la tabla
\definecolor{myblue}{RGB}{0,55,100}
\makeglossaries
\makeindex

\usepackage{rotating} %Para rotar texto, objetos y tablas seite. No se ve en DVI solo en PS. Seite 328 Hundebuch
                        %se usa junto con \rotate, \sidewidestable ....


%%%%%%%%%Formaro títulos y subtítulos%%%%%%%%%%%%%%%%
\usepackage{lmodern}  %*****Para Número GRANDE fancy en cada caítulo*****
\usepackage{microtype}
\usepackage{titlesec}
%\usepackage{graphicx}
\usepackage{sectsty}
\chapterfont{\color{myblue}}  % sets colour of chapters
\sectionfont{\color{myblue}}  % sets colour of sections
\subsectionfont{\color{myblue}}  % sets colour of subsections
\subsubsectionfont{\color{myblue}}  % sets colour of subsections


%%%Para cuadro con número en cada capítulo
\usepackage{titlesec}
\titleformat{\chapter}[block]
{\color{myblue}\normalfont\bfseries}
   {\rlap{\makebox[\linewidth][r]{%
      \rotatebox[origin=c]{90}{% %%Ángulo CAPÍTULO
      \normalfont\color{myblue}\large%
      \textls[90]{\textsc{\chaptertitlename}}%
   }\hspace{10pt}% %%%Espacio entre "CAPÍTULO y el cuadro
   {\setlength\fboxsep{0pt}%
      \colorbox{myblue}{\parbox[c][3.0cm][c]{2.5cm}{% %%Tamaño de cuadro
      \centering\color{white}\fontsize{80}{90}\selectfont\thechapter}%
   }}}%
   }}
   {0pt}
   {\huge\parbox{\dimexpr\linewidth-3.5cm}}[{\medskip\color{myblue}\titlerule[2.5pt]}\color{myblue}\vskip2pt\titlerule]
    \usepackage{lipsum}

%%%linewidth-3.5cm} espacio entre título de cápituloo y cuadro con el número
%%%vskip2pt espacio entre la linea pequeña y la grande

%%%%{\color{myblue}\normalfont\bfseries} color título
%*****FIN Para Número GRANDE fancy en cada caítulo*****

%%%%%%%%%FIN Formaro títulos y subtítulos%%%%%%%%%%%%%%%%

%%%******Formato para tablas*******
\usepackage{pgfplotstable, booktabs}
\pgfplotsset{compat=1.8}
\usepackage{longtable}
\usepackage{array}

%%%****************Formato título figuras************
% Define a new caption style
\DeclareCaptionFormat{bold}{\textbf{#1#2}\textbf{#3}}
\captionsetup[figure]{format=bold, labelfont=bf}

\DeclareCaptionFormat{colored}{\color{myblue}\textbf{#1#2}\textbf{#3}}
\captionsetup[figure]{format=colored, labelfont=bf}

%%%%%%Labels top para las subfiguras
\captionsetup{font={bf,small},skip=0.25\baselineskip}
\captionsetup[subfigure]{font={bf,small}, skip=1pt, singlelinecheck=false}
%%%****************FIN Formato título figuras************

%%%%%%% Para Lista de figuras
\DeclareCaptionFormat{bold}{\textbf{#1#2}\textbf{#3}}
\captionsetup[table]{format=bold, labelfont=bf}
\DeclareCaptionFormat{colored}{\color{myblue}\textbf{#1#2}\textbf{#3}}
\captionsetup[table]{format=colored, labelfont=bf}


\renewcommand{\theequation}{\thechapter-\arabic{equation}}
\renewcommand{\thefigure}{\textbf{\thechapter-\arabic{figure}}}
\renewcommand{\thetable}{\textbf{\thechapter-\arabic{table}}}


\pagestyle{fancyplain}%\addtolength{\headwidth}{\marginparwidth}
\textheight22.5cm \topmargin0cm \textwidth16.5cm
\oddsidemargin0.5cm \evensidemargin-0.5cm%
\renewcommand{\chaptermark}[1]{\markboth{\thechapter\; #1}{}}
\renewcommand{\sectionmark}[1]{\markright{\thesection\; #1}}
\lhead[\fancyplain{}{\thepage}]{\fancyplain{}{\rightmark}}
\rhead[\fancyplain{}{\leftmark}]{\fancyplain{}{\thepage}}
\fancyfoot{}
\thispagestyle{fancy}%


%\addtolength{\headwidth}{0cm}
%\unitlength1mm %Define la unidad LE para Figuras
%\mathindent0cm %Define la distancia de las formulas al texto,  fleqn las descentra
%\marginparwidth0cm
%\parindent0cm %Define la distancia de la primera linea de un parrafo a la margen

%Para tablas,  redefine el backschlash en tablas donde se define la posici\'{o}n del texto en las
%casillas (con \centering \raggedright o \raggedleft)
\newcommand{\PreserveBackslash}[1]{\let\temp=\\#1\let\\=\temp}
\let\PBS=\PreserveBackslash

%Espacio entre lineas o interlineado del documento
\renewcommand{\baselinestretch}{1.5}

%Neuer Befehl f\"{u}r die Tabelle Eigenschaften der Aktivkohlen
\newcommand{\arr}[1]{\raisebox{1.5ex}[0cm][0cm]{#1}}

%Neue Kommandos
\usepackage{Befehle}
%Inicio del documento. Tener en cuenta que hay archivos auxiliares

%%%Para los marcadores más adelante de las gráficas
\definecolor{falured}{rgb}{0.5, 0.09, 0.09} %%2.0%
\definecolor{emerald}{rgb}{0.31, 0.78, 0.47} %%%3.5%
\definecolor{goldenpoppy}{rgb}{0.99, 0.76, 0.0} %%5.0
\definecolor{deepmagenta}{rgb}{0.8, 0.0, 0.8}  %%7.0%

%%%%%%%Para hacer tabla Pad tipo usado en el Factor Lambda Chap 3
\usepackage{environ}
\newcommand{\TablePad}{\vspace{-0.2cm}}
\NewEnviron{EntriesTableDuration}{
\TablePad
\begin{tabularx}{\textwidth}{@{}p{0.10\textwidth}@{\hspace{0.02\textwidth}}p{0.88\textwidth}@{}}
  \BODY
\end{tabularx}
\TablePad
}
\NewEnviron{EntriesTableYear}{
\TablePad
\begin{tabularx}{\textwidth}{@{}p{0.05\textwidth}@{\hspace{0.01\textwidth}}p{0.94\textwidth}@{}}
  \BODY
\end{tabularx}
\TablePad
}
%%%%Fin formato%%%

%%%%Para símbolo de aprox igual
\DeclareFontFamily{U} {cmsy}{}
\DeclareFontShape{U}{cmsy}{m}{n}{
  <-6> cmsy5
  <6-7> cmsy6
  <7-8> cmsy7
  <8-9> cmsy8
  <9-10> cmsy9
  <10-12> cmsy10
  <12-> cmsy12}{}
\DeclareSymbolFont{Xcmsy} {U} {cmsy}{m}{n}
\DeclareMathSymbol{\Xapprox}{\mathrel}{Xcmsy}{25}
%%%Fin


%%Colocar "[numero]." punto después de número y espacio en bibliografía
\usepackage[numbers]{natbib}
\makeatletter
%\renewcommand{\@biblabel}[1]{\makebox[1em][l]{#1.}} Solo el número
\renewcommand{\@biblabel}[1]{[#1].}
\makeatother

%%%%Símbolo derivda parcial para monocapas
\DeclareMathSymbol{\partial}{\mathord}{letters}{"40}