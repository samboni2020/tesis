\chapter{Introducción}

\section{Membrana biológica}

La membrana celular, también llamada membrana plasmática es la barrera física entre la célula y el ambiente exterior de la misma.  Ésta tiene la característica de actuar como una estrucura semipermeable y esta compuesta principalmente de una gran variedad de lípido y proteína embedidas. Sin embargo, esta barrera física no solamente se limita a permitir la entra y/o salida de componentes en la célula, sino que tiene gran relevancia en varios procesos celulares como son la tracción celular y señalización celular \hl{cell-signalling}. 


Las propiedades mecánicas de la membrana, como la tensión superficial y el módulo de elasticidad estan extrechamente relacionadas a la función celular. Estas propiedades son moduladas por el citoesqueleto y una red de proteínas filamentosas (o escleroproteínas)  ancladas a la membrana vía proteínas integrales de membrana \cite{Vinogradova2002, Karagoz2021}.
%%https://theses.hal.science/tel-01685239/preview/KURAUSKAS_2017_archivage.pdf

\section{Integrinas}

Las integrinas son parte de una gran familia de receptores de membrana de paso único. Estan compuestas por dos subunidades, en cada subunidad se distinguen tres dominios, un gran dominio \ac{ec} (800 amino ácidos), una $\alpha$ hélice \ac{tm} (20 amino ácidos) y un corto dominio \ac{ic} o citoplasmático mayoritariamente no estructurado (ref). Se conocen cerca de 25 miembros de integrinas (18 subunidades - $\alpha$  y 8 subunidades - $\beta$) (ref), cada miembro tiene sitios de reconocimientos diferentes a unión de ligandos, asi como también interaccionan con diferentes proteínas de adhesión.


Las integrinas son receptores fisiológicamente relevantes que median la interacción celular controlada por ligandos y mensajes intra- e intercelulares, transducen señales bidireccionalmente entre el interior y el exterior celular.


Podemos observar la secuencia primaria correspondiente al péptido $\alpha$IIb desde N al C-terminal que incluye un pequeño loop extracelular (10 resíduos), en rojo se resalta el segmento o dominio \ac{tm} (28 resíduos) y la cola citoplamática, en \textbf{negrilla} se resalta región con estructura secundaria $\alpha$-hélice,  de forma similar en la Figura x se aprecia la secuencia primaria para el péptido $\beta$3


Como se observa $\beta$3 tiene un segmento \ac{ic} mucho más largo que $\alpha$IIb
%%Acá hablan de una base de datos de solo TM \url{https://onlinelibrary.wiley.com/doi/epdf/10.1002/pro.4318}
%%y hay info del tilt angle


\section{Motivación y objetivo de la tesis}